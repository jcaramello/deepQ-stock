\chapter{Conclusiones}

Durante este capitulo final, queremos comentar algunas observaciones y/o recomendaciones que creemos es necesario resaltar.
Recordemos que el objetivo inicial del proyecto era investigar la posible aplicabilidad y efectividad del uso de reinforcement learning y deep learning en el desarrollo de sistemas de trading. Luego de varios días de desarrollo y pruebas, y que mas allá de los resultados obtenidos, que tal vez estén lejos de igualar o sustituir a un inversor experimentado, el sistema demostró ser fiable en el sentido de es posible implementar un algoritmo que solamente evaluando la evolución del precio de un activo, aprenda a tomar decisiones sobre el.\\ Podemos no estar conforme con las ganancias obtenidas, pero es claro que es posible generar un proceso de aprendizaje, el cual le permite al agente mejorar su desempeño a lo largo del tiempo, como así también poder repetirlo en simulaciones siguientes.
\\\\
También, debemos reconocer aquí nuestra falta de experiencia y comprensión en el diseño y desarrollo de redes neuronales, el cual es posible que haya podido influenciar en la performance del agente. O que tal vez, el uso de una red neuronal no sea la mejor solución para estimar la función de valor $Q(s, a)$, en cualquier caso, creemos que el potencial de mejora en este aspecto es enorme, y que junto con ello, la posibilidad de mejorar el desempeño de nuestro a agente.\\

Por otro lado, queremos resaltar, que no nos resulto sencillo determinar la efectividad de la performance del agente, es decir, no fue sencillo determinar cuando un trade es bueno o no, y en particular poder determinar si una decisión de compra o venta, es correcta o no. Tal vez también, producto de nuestra falta de conocimiento y/o entendimiento de los mercados.Y por lo tanto, también, en este sentido, creemos que hay mucha posibilidad de mejoras, y con ello poder modificar la señal de recompensa para tratar de mejorar el proceso de aprendizaje del agente.\\

En definitiva, podemos decir que los resultados son prometedores, y que como prueba de concepto ha logrado satisfacer nuestras expectativas, lo que nos permite pensar que en un futuro no muy lejano, este tipo de sistemas sean mas comunes ya sea como soporte a inversores o como sistemas completamente automatizados que permitan la administración de una cartera de inversiones.