\chapter{Introducción}

Este documento forma parte del proyecto final de carrera y se acompaña junto con el código del agente desarrollado, disponible también \href{https://github.com/jcaramello/deepQ-stock/}{\textcolor{blue}{online}}. En este documento se pretende dar una presentación formal a los resultados de investigación obtenidos durante el desarrollo del proyecto, como así también una descripción del problema a resolver y de la solución adoptada, detallando cuales fueron cada una de las decisiones de diseño adoptadas y por que se adoptaron.

\section{Alcance del proyecto}
El proyecto se planteo como un trabajo de investigación que permitiera analizar cuales son las herramientas que brinda el machine learning para desarrollar un agente capaz de invertir en activos financieros(acciones, bonos, obligaciones negociables, etc), en particular, a través  del uso de reinforcement learning.

El proyecto pretende ser una prueba de concepto, que permita abordar el desarrollo de sistemas de trading utilizando reinforcement learning. En particular, se buscara identificar cada uno de los componentes que plantea el framework de reinforcement learning, de forma tal, de tratar de encontrar una representación adecuada.
\\
A su vez, durante el proceso de investigación se tratara de comprender la complejidad y las problemáticas propias del dominio del problema, en este caso, los mercados financieros y el trading en ellos, las cuales sera necesario comprender, para poder modelar una solución mas eficiente.
\\
Queda fuera del alcance de este proyecto lo siguiente:

\begin{itemize} % lista con viñetas
	\item Desarrollar un nuevo algoritmo de RL
	\item Desarrollar un agente capaz generar ganancias
	\item Desarrollar una comparación de diferentes implementacion del RL.
    \item Cubrir todas las peculiaridades de un mercado financiero.
\end{itemize}

\section{Objetivos}
A continuación se detallan los objetivos del proyecto

\subsection{Objetivo general}
Investigar la posible aplicabilidad de reinforcement learning en el desarrollo de sistemas de trading que permitan optimizar y automatizar la toma de decisiones de un potencial inversor.

\subsection{Objetivos específicos}

\begin{itemize} % lista con viñetas
	
	\item Brindar una descripción general del funcionamiento de los mercados financieros.
	\item Brindar una descripción general de Reinforcement Learning.
	\item Brindar una especificación detallada de cómo modelar el problema de trading utilizando los elementos que propone el framework de RL.
	\item Desarrollar un entorno que permita realizar la simulación de un mercado financiero 
	\item Desarrollar un agente inteligente capaz de percibir su entorno y tomar decisiones de compra o venta de un activo financiero
\end{itemize}

\section{Metodología}
En primer lugar se definirá el tópico central de la investigación, junto con los alcances y limitaciones del proyecto.
En una segunda etapa, se hará una investigación sobre el funcionamiento de los mercados financieros, en particular, se buscará entender el funcionamiento del análisis técnico de los mercados  y/o de los principales indicadores bursátiles, de forma tal de poder capturar estos conceptos y poder modelarlos en el desarrollo del agente.  
En tercer lugar se llevará a cabo una investigación de los diferentes algoritmos de RL y de cómo modelar el problema de trading utilizando los elementos que propone el framework.
Por último, se proseguirá con el desarrollo del agente inteligente, de forma tal de poder integrar todo el conocimiento adquirido en las etapas previas. 

\section{Contenido}
Por último, concluimos este capitulo introductorio con una breve descripción de cada uno de los capítulos siguientes

\begin{itemize} % lista con viñetas
	\item Capitulo 2 - Mercados Financieros: En este capitulo se presenta el problema del trading, se dará una breve explicación del funcionamiento de un mercado financiero y de como operan los inversores en el, abarcado conceptos como acciones, precio, análisis técnico vs análisis fundamental, drivers, indicadores bursátiles, tendencias, etc.
	\item Capitulo 3 - Reinforcement learning y deep learning: En este capitulo de introducirá brevemente el framework de reinforcement learning y cada uno de sus componentes, para luego esbozar la representación que se adoptara junto con un pseudo algoritmo que mostrara una primera aproximación a la solución elegida
	\item Capitulo 4 - Arquitectura y Diseño: Aquí se hará una descripción detallada de la arquitectura y el diseño del sistema implementado, se mostraran detalles y demás peculiaridades de implementacion, en particular parámetros de configuración, funcionalidades, etc. Se mostraran algunas las partes mas relevantes del código del agente.
	\item Capitulo 5 - Evaluación y desempeño: En esta sección analizaremos el desempeño y eficacia de la solución implementada, mostrando y analizando los diferentes resultados obtenidos durante la ejecución del agente.
	\item Capitulo 6 - Conclusiones y Recomendación: Como punto final, esta sección estará destinada a comentar diferentes conclusiones que pudimos establecer, junto con algunas recomendaciones y/o observaciones de posibles mejoras a futuro.
	\item Capitulo 7 - Bibliografía y Referencias: Listado de las diferentes referencias que formaron parte de la investigación.
\end{itemize}