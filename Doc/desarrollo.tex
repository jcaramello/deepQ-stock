%documento con el cuerpo del informe.
%puede ser un solo archivo o varios, por ejemplo uno por cada capítulo

%- capítulo ---------------------------------------------------------
\chapter{Introducción}
Este \emph{documento de muestra}, es un ejemplo del uso de la clase \LaTeX~ \texttt{eieproyecto} para escribir el informe del curso IE0499 - Proyecto Eléctrico y puede ser utilizado como base para la preparación del informe.

El archivo \texttt{eieproyecto.cls}, debe incluirse en la carpeta donde se encuentran los demás archivos utilizados para la confección del informe.

Si se está leyendo la versión en formato \texttt{.pdf} de este documento, se recomienda tener a mano el archivo original \texttt{.tex}, para ver la información incluida en las líneas de comentario (iniciadas con \texttt{\%}), así como los comandos \LaTeX~ utilizados para su elaboración.

Todo el informe del proyecto debe escribirse en \emph{pasado impersonal}, siguiendo las instrucciones generales dadas en este documento, las cuales se complementan con el \emph{Manual de usuario de la clase} \texttt{eieproyecto} \citep{vmaeieproyman}.  El documento debe ser conciso y objetivo. 

Debe cuidarse la redacción del informe, no solo respecto a la ortografía, si no también en cuanto a la estructura gramatical y la puntuación, la cual debe ser conforme a las reglas gramaticales del español.

Debe hacerse uso de las unidades del Sistema Internacional (SI) \citep{RTCR443-2010}, recordando que debe emplearse la coma (,), como separador decimal.

El informe debe escribirse utilizando \LaTeX~ y sus \emph{aspectos de forma} (``el formato''), están predefinidos por la clase \texttt{eieproyecto} y no deben modificarse.

Información general sobre el uso de \LaTeX~ se puede encuentran en el folleto de \cite{nsces}, en forma más detallada en el libro de \cite{latexcomp} y en \emph{CervanTeX - Grupo de Usuarios de \TeX~ Hispanohablantes} (\url{http://www.cervantex.es/}).  Además, en el sitio web del curso se encuentra material de referencia complementario.


%- sub sección ------------------------------------------------------
\subsection{Portada}
Para la portado del informe, se debe indicar la siguiente información:
\begin{itemize}
	\item Título del proyecto (\texttt{title}).
	\item Nombre completo del estudiante \texttt{(autor}).
	\item Fecha de la presentación en formato ``Mes de año'' (\texttt{date}).
\end{itemize}

El título debe reflejar y en lo posible destacar, el aspecto más importante del proyecto.  Este debe ser corto, se sugiere que de no más de 10 a 12 palabras y, en todo caso, no puede exceder 15 palabras.

En el título, solo su primera letra, la de los nombre propios y la de los acrónimos de más de cuatro letras, estará en mayúscula, así como los acrónimos de cuatro letras o menos.

\subsection{Hoja de aprobación}
Se debe indicar la conformación del Tribunal evaluador, para la Hoja de aprobación del proyecto, proporcionando:
\begin{itemize}
	\item Grado y nombre del profesor guía (\texttt{dca}).
	\item Grado y nombre de los miembros lectores (\texttt{maca} y \texttt{mbca}).
\end{itemize}

Para indicar el grado académico de los miembros del Tribunal, se utilizará antes del nombre la abreviatura Ing. o Ing.$^a$ (bachillerato o licenciatura en ingeniería), M.Sc., Dr., Dra., Lic., Lic.$^a$ , u otro según corresponda.  También se puede utilizar después del nombre, la abreviatura M.Sc. o Ph.D., según corresponda.

\subsection{Resumen}
Este debe describir, en forma sucinta, los objetivos, el trabajo realizado, los resultados principales y las conclusiones del proyecto, y no debe exceder una página.

Se recomienda que el resumen sea escrito, después que se haya completado la elaboración del borrador final del informe.

\subsection{Índices}
Después del resumen se incluirá el índice general  del informe (\texttt{tableofcontents}) y luego de este, si fueran necesarios, el índice de figuras (\texttt{listoffigures}) y el índice de cuadros (\texttt{listoftables}).

\subsection{Nomenclatura}
Todos los símbolos y acrónimos utilizados en las ecuaciones y el texto del informe, deben listarse en orden alfabético en la Nomenclatura.

La nomenclatura es una lista con descripción (\texttt{description}).  En esta los símbolos matemáticos deben escribirse utilizando \texttt{\$símbolo\$}.

Los acrónimos, además de incluirse en la nomenclatura, deben describirse en el texto, pero solo la primera vez que se utilizan.

%--------------------------------------------------------------------
\section{Introducción del informe}


%- declaración de un capítulo ---------------------------------------
\chapter{Mercados Financieros} \label{sec:L02}
En el segundo capítulo del informe, debe resumirse el estudio realizado sobre \emph{estado de la técnica}, en la temática relacionada con el proyecto.  Este se puede denominar ``Antecedentes'', ``Marco de referencia'', ``Base teórica'', o ``Marco teórico''.
 
%- declaración de una sección ---------------------------------------
\section{Ecuaciones}
Las ecuaciones estarán centradas y numeradas en forma secuencial por capítulo, al margen derecho.  La referencia a ellas se hará utilizando su número.

¡Texto de ejemplo! - ``El modelo utilizado para representar al proceso, es de primer orden más tiempo muerto, dado por la función de transferencia

\begin{equation}  %inclusión de ecuaciones
	P(s) = \frac{K \me^{-Ls}}{Ts+1}, \label{ec:01}
\end{equation}

\noindent donde $K$ es la ganancia, $T$ la ...''  
%si el texto después de la ecuación no inicia un nuevo párrafo y se ha insertado una linea en blanco depues de esta, es necesario poner \noindent para que el texto siguiente no tenga sangría (formato predeterminado).

Las ecuaciones forman parte del texto, por lo que deben terminarse con el signo de puntuación requerido, una coma o un punto.

Para referirse a ellas se hace uso de la etiqueta (\texttt{label}) asignada a la ecuación usando \texttt{\textbackslash eqref\{etiqueta\}} que mostrará su número.  Por ejemplo ``El modelo \eqref{ec:01} es el más utilizado para ...''

Usando \texttt{equation}:
\begin{equation}
	\tau \frac{\md T_{tc}(t)}{\md t} + T_{tc}(t) = T_{gas}(t).
\end{equation}

Ecuaciones alineadas utilizando \texttt{align}:

\begin{align}
	L_1 \frac{\md i_{L_1} (t)}{\md t} &= v(t) - R_1 i_{L_1}(t) - v_c(t), \\
	C \frac{\md v_c (t)}{\md t} &= i_L(t)- \frac{1}{R_2} v_c(t).
\end{align}

\section{Figuras y cuadros}
Las figuras y los cuadros son \emph{elementos flotantes}. Aunque se le puede ``sugerir'' a \LaTeX~ donde ubicarlos, es conveniente dejarlos ``flotar''.

\subsection{Figuras}
Las referencias a las figuras debe hacerse utilizando el número asignado a ellas.  Para esto se le asigna una etiqueta (con \texttt{label}) y luego se utiliza esta para hacer la referencia (con \texttt{ref}).  Usar en el texto el término ``figura'' y no Fig.'' o ``fig.''.

La leyenda (con \texttt{caption}) de la figura, irá en la parte inferior de la misma.  Como en forma predeterminada en la clase \texttt{eieproyecto} las figuras están centradas, no es necesario usar \texttt{centering} para hacerlo.

Por ejemplo ``Considérese el diagrama de bloques mostrado en la figura \ref{fig:01} en donde el proceso controlado está dado por ...''.

No utilizar ``... en la siguiente figura ...'', emplear siempre el número correspondiente para referirse a ellas.

%inclusión de una figura (debe estar en formato .eps si se usa latex)
\begin{figure}
	%\includegraphics[width=0.7\linewidth]{lc_1gl}
\caption{Sistema de control realimentado.} \label{fig:01x}
\end{figure}

Cuando las figuras son muy pequeñas, se puede colocar la leyenda al lado de la misma, con el ambiente \texttt{SCfigure} del paquete \texttt{sidecap}.  Un ejemplo de esto se muestra en la figura \ref{fig:01}.

\begin{SCfigure}
	%\includegraphics[width=0.65\linewidth]{lc_1gl}
\caption{Lazo de control de un proceso de una entrada y una salida.} \label{fig:01}
\end{SCfigure}

Cuando un gráfico muestre varias curvas, estas deben poderse distinguir, no solamente en la pantalla de la computadora, usando diferentes colores, si no también en una impresión en blanco y negro, utilizando lineas de trazos diferentes, como se muestra en la figura \ref{fig:respeustas}.

\begin{SCfigure}
	%\includegraphics[width=0.65\linewidth]{prueba01}
\caption{Respuesta del circuito simulado, para dos valores de $R_5$.} \label{fig:respeustas}
\end{SCfigure}

\LaTeX~ nunca coloca las figuras y los cuadros en una página anterior a la en que son incluidas.  Los elementos flotantes los coloca en la página donde se hace referencia a ellos, o en una de las siguientes.

Además, en el texto debe hacerse referencia a todas las figuras y cuadros incluidos en el informe.  Si alguno de ellos no se menciona en el texto, es que no se requiere para entender el desarrollo presentado y por lo tanto es innecesario y se podría omitir sin que se afecte el informe.

\subsection{Cuadros}
Los cuadros son el otro elemento flotante utilizado en los informes y también es conveniente dejar que \LaTeX~ los coloque en donde considere que es más adecuado.

Los cuadros no llevarán ninguna línea divisoria vertical, solo horizontales. Una en la parte superior (\texttt{toprule}), una bajo la línea de cabecera (\texttt{midrule}) y una en la parte inferior (\texttt{bottomrule}).  Normalmente basta con estas tres líneas, pero si fuera necesaria alguna otra para una división horizontal, esta debe ser del tipo \texttt{midrule}.

Se recomienda revisar los comandos para la construcción de cuadros, incluidos en el manual de la clase \texttt{memoir} \citep{memoir2011}, o en la del paquete \texttt{booktabs} \citep{fear2005}.

La leyenda (\texttt{caption}) del cuadro se mostrará en la parte superior.  Para poder referirse al cuadro (con \texttt{ref}), se le asigna una etiqueta (con \texttt{label}).

En forma predefinida, los cuadros se mostrarán centrados horizontalmente, por lo que no es necesario hacer esa indicación. 

El cuadro \ref{tab:01} es un ejemplo de un cuadro de datos simple.

%inclusión de un cuadro con datos
\begin{table}
\caption{Parámetros de los modelos.} \label{tab:01o}
		\begin{tabular}{@{}*{4}{c}@{}}
    \toprule
    $K_p$ & $T_1$ & $T_2$ & $L$ \\
    \midrule
     1,01 & 1,50 & 0,75 & 0,12 \\
		 1,15 & 2,37 & 0,15 & 0,28 \\
		 2,25 & 5,89 & 2,15 & 1,60 \\
    \bottomrule
    \end{tabular}
\end{table}

Si la primera columna corresponde a leyendas o parámetros que identifican los datos de la línea, esta debe estar justificada a la izquierda, como se muestra en el cuadro \ref{tab:AH}, que ha sido tomada de \cite{astromhagglund2006}.

\begin{table}
\caption{Parámetros de los controladores ...} \label{tab:AH}
\begin{center}
    \begin{tabular}{@{}l*{7}{c}@{}}
    \toprule
    Controller & $K$ & $K_i$ & $K_d$ & $\beta$ & $T_i$ & $T_d$ & IAE \\
    \midrule
    PD &  1,333 & 0 & 1,333 & 1 & 0 &1 & $\infty$ \\
		PI & 0,433 & 0,192 & 0 & 0,14 & 2,25 & 0 & 6,20 \\
		PID MIGO & 1,305 & 0,758 & 1,705 & 0 & 1,72 & 1,31 & 2,25 \\
		PID $T_i=4 \ T_d$ & 1,132 & 0,356 & 0,900 & 0,9 & 3,18 & 0,80 & 2,51 \\
    \bottomrule
    \end{tabular}
\end{center}
\end{table}

Se puede especificar una cabecera para más de una columna y utilizar lineas horizontales que abarquen solo unas pocas columnas, como se muestra en el cuadro \ref{tab:muestra}.

\begin{table}
\caption{Ejemplo de otro cuadro.} \label{tab:muestra}
	\begin{tabular}{@{}l*{4}{c}@{}}
	\toprule
	& \multicolumn{2}{c}{Prueba 1} & \multicolumn{2}{c}{Prueba 2} \\
	\cmidrule(l{2pt}r{2pt}){2-3}\cmidrule(l{2pt}r{2pt}){4-5} 
	& $\Delta E=5$ V & $\Delta E = -5$ V & $\Delta E = 10$ V & $\Delta E = -10$ V \\
	\midrule
	Ganancia             &  1,06 & 0,98 & 1,12 & 0,97 \\
	Tiempo subida, s  &  5,67 & 5,89 & 6,02 & 5,74 \\
	Sobrepaso máx, \%        &  2,67 & 3,25 & 2,91 & 1,56 \\
	Error, \% &  0,25 & 0,56 & 0,97 & 0,18 \\
	\bottomrule
	\end{tabular}
\end{table}

\newpage
Cuando los cuadros son pequeños (abarcan menos de la mitad del ancho del texto), se puede colocar la leyenda a la par del cuadro, utilizando el ambiente \texttt{SCtable} del paquete \texttt{sidecap}, tal como se muestra en el cuadro \ref{tab:01}.  Compare este, con el cuadro \ref{tab:01o}.

\begin{SCtable}
\caption[Parámetros de los modelos]{Parámetros de los modelos, obtenidos a partir de las tres curvas de reacción.} \label{tab:01}
    \begin{tabular}{@{}*{4}{c}@{}}
    \toprule
    $K_p$ & $T_1$ & $T_2$ & $L$ \\
    \midrule
     1,01 & 1,50 & 0,75 & 0,12 \\
		 1,15 & 2,37 & 0,15 & 0,28 \\
		 2,25 & 5,89 & 2,15 & 1,60 \\
    \bottomrule
    \end{tabular}
\end{SCtable}

%------------------------------
\chapter{El Problema}
El capítulo 3 y los subsiguientes (si fueran necesarios), mostrarán el trabajo realizado en el proyecto, por lo que su cantidad, títulos y divisiones, se dejan a discreción del estudiante, con la aprobación del profesor guía y demás miembros de Tribunal evaluador.

\chapter{Arquitectura y Diseño}



%-----------------------------

\chapter{Evaluación y Desempeño}



%-----------------------------
\chapter{Conclusiones y Recomendaciones}
El informe debe terminarse con la enumeración de las principales conclusiones derivados del trabajo realizado.  En particular, debe verificarse el cumplimiento de los objetivos planteados para el mismo.

\subsection{Conclusiones}
El aporte (\emph{novedad}) hecho con el proyecto, debe destacarse.

Las conclusiones pueden enumerarse en forma suscinta como una lista, ya sea itemizada o numerada.

\subsection{Recomendaciones}
Con base en las trabajo realizado y las conclusiones sobre el mismo, puede ser necesario incluir una sección, o lista, de recomendaciones.  Por ejemplo, sobre la utilización de otro enfoque para resolver el problema.


\chapter{Bibliografía}