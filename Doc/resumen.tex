%Archivo con el texto del resumen
% este no debe exceder una página
\begin{center}\huge{\textbf{Resumen}}\end{center}

%--------------------------------------------------------------------
\noindent
\\
\\
Vivimos en un mundo cada vez más digitalizado donde el acceso a la información es cada vez más fácil y abundante, un mundo que en los últimos años se ha transformado radicalmente y que sin duda lo seguirá haciendo en los años venideros. Los avances en machine learning, han permitido automatizar tareas, que hasta hace algunos años parecía imposible, en la actualidad podemos encontrar múltiples ejemplos de ellos, como por ejemplo, waymo(Google self-driving car), los sistemas de recomendación como los usados por netflix o mercado libre, o Deep Mind.
\\
\\
Uno de los desafíos más interesantes en este área, es el desarrollo de sistemas de trading que permitan automatizar la comercialización de activos financieros, con el fin de administrar eficientemente un porfolio de inversiones.
El siguiente trabajo esta inspirado en Deep Mind \cite{8}, un agente que combinando reinforcement learning y deep learning aprende a jugar juegos de Atari. \\
En este paper buscaremos investigar la aplicabilidad y efectividad de las técnicas utilizadas en deep mind, en el desarrollo de sistemas de trading. 


