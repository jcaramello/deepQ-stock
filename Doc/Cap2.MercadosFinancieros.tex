\chapter{Mercados Financieros}

\section{Introducción}
En economía, un mercado financiero es un espacio (físico, virtual o ambos) en el que se realizan los intercambios de instrumentos financieros y se definen sus precios. Los mercados financieros están afectados por las fuerzas de oferta y demanda. La clave del éxito está en saber predecir el futuro y actuar en consecuencia. Quedarse largo en una posición (comprar) si se piensa que el mercado va a subir, o deshacer posiciones o quedarse corto si se piensa que el mercado va a bajar. Si se consigue hacer esto en forma reiterada se podrán obtener ganancias y con ello incrementar el capital inicial.

\section{Análisis Técnico y Análisis Fundamental}
Para intentar saber cómo va a estar un valor en el futuro, se distinguen tradicionalmente dos corrientes bien diferenciadas: los que siguen el análisis técnico y los que siguen el análisis fundamental. Los fundamentales se basan en que el valor de una acción esta dado por los beneficios futuros de la empresa. Ni más, ni menos. Lo que intentan es determinar cuáles serán esos beneficios futuros, y para ello tratan de conocer diferentes detalles de la empresa: noticias que les afecten, posibles movimientos societarios, estrategias, competidores, nuevos productos, etc. Toda la información micro económica tiene impacto en dichos beneficios futuros. Así como también la macro económica: cómo evoluciona el entorno general de la empresa, el entorno regulatorio, el entorno político, etc. Se trata, en definitiva, de analizar la mayor cantidad de información posible, y de convertir esa información en cuentas de resultados provisionales que se puedan descontar para hallar el valor actualizado de la acción. 
El análisis técnico, por el contrario, se basa en que el precio de la acción lo descuenta todo. Es decir, todos los factores relevantes a la inversión, cualquiera que ellos sean, pueden ser reducidos al nivel de precios de la acción y volumen transado. El precio de mercado representa el total conocimiento de los inversionistas respecto de cualquier activo dado en un momento particular. Además, refleja todas las noticias sobre el mercado así como la suma de conocimientos de los participantes en éste.  Aquí se habla de tendencias alcistas o bajistas, de líneas de soporte (cotizaciones donde se cree que la acción dejará de bajar y tendrá un "rebote"), líneas de resistencia (cotizaciones en las que el valor de la acción se atascará y que le costará "romper"), etc. 
La lógica nos dice que el análisis fundamental es el que tiene más sentido. Sin embargo, en la realidad esto no siempre es así. Lo cierto es que cuanta más gente crea en los análisis técnicos, más probabilidad tendrán de ser reales sus predicciones (ya que la gente actuará como si fueran reales, contribuyendo a su efectiva realización).
\\\\
Este proyecto se basa fundamentalmente en las ideas del análisis técnico y las herramientas que este brinda, las cuales, serán los pilares fundamentales sobre los que el agente tomara sus decisiones.
Para comprender un poco mas acerca de este y visualizar como es posible tomar decisiones acertadas solamente apoyándonos en el análisis técnico, debemos mirar mas en detalles, algunos aspecto de la teoría clásica financiera.

\section{Teoría Clásica}

Gran parte de la teoría financiera clásica parte del principio fundamental que los inversores son racionales y que los precios del mercado reflejan en todo momento y de manera instantánea el valor fundamental de los títulos.
Este principio fundamental establece que la competencia entre los distintos participantes que intervienen en el mismo, conduce a una situación de equilibrio en la que el precio de mercado de un activo constituye una buena estimación de su precio teórico, es decir, que los precios que se negocian en el mercado reflejan toda la información existente y se ajustan total y rápidamente a los nuevos datos que puedan surgir. La consecuencia de este principio es que un inversor racional no puede hacer nada para “ganar” al mercado.\\	

De acuerdo con este principio de racionalidad económica, lo que debe hacer un inversor es intentar maximizar su riqueza final. Para lograr esto, lo mejor que puede hacer este inversor racional es invertir en el mercado de manera diversificada de una forma igual a la del mercado y permanecer en esta misma cartera salvo por necesidades de liquidez o debido a variaciones de su situación actual o de cambio en sus necesidades futuras.
\\\\
A pesar de la cantidad de libros de texto y de artículos que sostienen los principios anteriores sobre la forma en que deben comportarse los inversores, lo cierto es que la evidencia empírica nos dice que las cosas no suceden de la forma en la que deberían suceder según este principio,o por lo menos, no enteramente.
Para poder explicar este fenómeno deberemos detenernos en algunos aspectos fundamentales en el proceso de decisión de los inversores. 
\\
Un claro ejemplo de esto podemos verlo si analizamos el proceso de decisión de venta, de acuerdo con la teoría clásica, los precios de cualquier activo siguen un movimiento aleatorio. Esto significa que la mejor predicción sobre el precio futuro es la que se tiene hoy. La consecuencia inmediata de esto es que no tiene ningún sentido vender, para a continuación, volver a comprar éste mismo activo u otro diferente.
Dado que las expectativa de ganancia debido a la diferencia entre los precios venta y de compra sería nula. Solo tendríamos una pequeña pérdida debido al coste de la transacción.
En otras palabras, el mercado no es predecible y, en consecuencia, no es posible obtener un beneficio realizando trading. Sin embargo, veamos algunos conceptos que contradice esto:

\begin{itemize}
	\item \emph{La creencia de los inversores en la reversión a la media}, es el principio según el cual existe un valor medio de cada acción al cual se acaba volviendo en algún momento. Así, si un valor tiene un precio que el inversor cree que está por debajo del que le corresponde (su valor “medio”) , tarde o temprano, el precio de esta acción subirá hasta llegar a ese precio. Y, en consecuencia, recuperará las pérdidas que está teniendo en este momento. Lo mismo puede decirse cuando el valor esta por encima del valor medio. Así, si el inversor tiene una serie de valores que entiende que están infravalorados por el mercado, tiende a mantenerlos esperando que vuelvan a su valor “medio”.  	

	\item \emph{La aversión a la pérdida. } Si por alguna razón, tenemos alguna predicción confiable que nos dice que el precio de un activo va a bajar, lo racional es vender, independientemente de que con el activo a vender se obtenga un beneficio por su venta o no.
	La realidad nos muestra que esto no siempre es así, y que en general, los inversores tienden a no realizar las pérdidas, es decir, a no efectuar la venta real, en valores en los que pierden y, en cambio, vender antes de tiempo aquellos en lo que tienen ganancias. Una gran parte de los inversores venden valores ganadores y mantienen los perdedores.
	
	\item \emph{El efecto disposición}. El comportamiento debido a la aversión a la pérdida es una parte de un comportamiento más general de los inversores, llamado efecto disposición según el cual, los inversores mantendrían demasiado tiempo activos en pérdidas y venderían demasiado pronto activos con ganancias. Esto se da debido a que los inversores son mucho más sensibles a las pérdidas que a las ganancias en el sentido de que las primeras influyen el doble que las segundas.
	
	\item \emph{El efecto atención}. Los inversores tienden a invertir en aquellos valores que llaman mas su atención por la razón que sea, incluso aunque esta atención sea debida a noticias negativas, y como consecuencia, gran parte de los inversores diversifica mucho menos de lo que debería.
\end{itemize}

Estos argumentos y otros, nos indican que el inversor no siempre se comporta de manera esperada. Tal vez, el principal error de la teoría clásica sea partir de que los inversores son entes racionales y que no operan según sentimientos, euforia, miedo, avaricia o codicia.

\section{Fundamentos del análisis técnico}

Si bien es cierto que existen diversos argumentos en contra del análisis técnico, como por ejemplo:

\begin{itemize}
	\item La profecía del auto cumplimiento
	\item El pasado no sirve para predecir el futuro
	\item El paseo aleatorio
	\item Mercados Eficientes
\end{itemize}

Lo interesante aquí es entender que el análisis técnico no trata de predecir el valor futuro del precio de un activo, sino mas bien de responder la siguiente pregunta:\\\\

\emph{¿cómo cree el conjunto de inversionistas que evolucionará el precio en el futuro? }
\\

Es importante notar, que lo que importa no es la evolución del precio, sino lo cree \emph{la masa de inversores sobre la evolución del precio}, y ¿Por que esta pregunta es relevante?. Pues por que en el corto y mediano plazo, los precios se mueven mas por cuestiones psicológicas de los inversores que por variables financieras.\\

El análisis técnico nos dará las herramientas necesarias poder analizar desde un punto de vista estadístico, cual fue el comportamiento de la masa de inversores en el pasado, ante una situación similar, para luego poder tomar la decisión mas conveniente.
\\\\
En otras palabras lo que se busca es reducir el nivel de incertidumbre que tiene el inversor a la hora de decidir si compra o vende un activo, haciendo uso de diferentes herramientas matemáticas como las medias móviles, lineas de tendencias o patrones con el objetivo de determinar cual es el comportamiento mas probable de la masa de inversores.


